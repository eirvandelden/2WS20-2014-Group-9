\chapter{Conclusie}\label{cha:discussion}

Wij hebben in dit verslag een kansmodel beschreven, dat gebruikt maakt van twee random variabelen. Dit model hebben we verfijnd zodanig dat we het kunnen gebruiken binnen de context van de krasactie. We hebben vervolgens een analyse uitgevoerd met als resultaat een verdeling voor de relevante type loten en we hebben dit verslag besloten met een zekerheidsanalyse.\\

In hoofdstuk 3 hebben we de hoeveelheden loten van het type $\alpha, \beta, \gamma, \delta, a,b,c$ en $d$ vastgesteld die we per week verspreiden. Omdat we de beslissing hebben genomen dat we de loten elke week met een identieke verdeling zullen verspreiden, mogen we deze waarden simpelweg met 14 vermenigvuldigen om de verdeling van het totaal aantal lotente bepalen. In 14 weken moeten er 42 loten van het type $\alpha$, 70 van het type $\beta$, 84 van het type $\gamma$, 84 van het type $\delta$, 168 van het type $a$, 168 van het type $b$ en 168 van het type $c$ uitgedeeld worden. De overige 41216 loten zijn van type d, waarmee dus niks te verdienen valt.\\

Uitgaande van de verspreiding van loten die hierboven gegeven wordt met een vaste $p = 0.83$ kan het ED met 95$\%$ zekerheid zeggen dat het budget binnen \euro12.300 plusminus \euro922.5 blijft. Dit geeft dus een behoorlijk grote zekerheid en we vinden dit een acceptabele conclusie die we de opdrachtgever mogen mededelen. \\

Verder biedt deze verspreiding van loten ook zekerheid rondom een variabele $p$. Als $p$ lineair stijgt brengt dit maar een verschil van \euro40 met zich mee en dit is zeer laag. De echte situatie kan natuurlijk verschillen van een lineaire stijging. Echter zal het altijd goed te benaderen zijn door een constante lineaire stijging of de vaste waarde van $p$=0.83. In feite is het onderzoek naar het gedrag van $p$ van een hele aard. Dit heeft namelijk te maken met het gedrag van mensen en is vooral interessant vanuit een psychologisch perspectief.\\

We sluiten af met te zeggen dat onze uitvoering van dit probleem een van velen is en dat het niet per se de beste hoeft te zijn. Het zou interessant zijn om bijvoorbeeld een pure simulatie te bekijken en de resultaten te vergelijken met ons mathematische model. Dit zou als vervolgonderzoek kunnen worden gedaan.


