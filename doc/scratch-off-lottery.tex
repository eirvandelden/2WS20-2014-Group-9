\chapter{Opdrachtbeschrijving}\label{cha:scratch-off-lottery}

\section{Opdrachtbeschrijving}

Bij het Eindhovens Dagblad (ED) wil men een knetterende, stimulerende krasactie organiseren
voor haar krantenbezorgers.
Het gaat om krantenbezorgers die gedurende 14 weken, 6 dagen per week ’s ochtends
de krant bezorgen. Zij krijgen op de door hen te bezorgen stapel kranten een
masker. Dit is een uitdraaivel waarop aangegeven wordt op welke adressen ze de krant
moeten bezorgen en of er in hun wijk de dag ervoor klachten waren of dat ze klachtenvrij
bezorgd hebben.\\

Gewoonlijk spaart de bezorger voor elke klachtenvrije bezorgdag
stapelpunten (een soort airmiles/tankzegels) waarmee ze cadeaus kunnen bestellen uit
een spaarcatalogus). Dit spaarsysteem loopt tijdens de actie gewoon door.
Voor de speciale krasactie krijgt de bezorger half december voor 14 weken 14 krasloten
met op elk weeklot 6 krasvakjes. Voor elke klachtenvrije dag mogen ze 1 willekeurig vakje
openkrassen. Voor onder de kraslaag wil het ED uitgaan van 6 verschillende symbolen,
nl. een ster, een krant, een brievenbus, een fiets, een krantentas en een hond. De
bezorgers kiezen dus zelf welke hokjes ze openkrassen voor een klachtenvrije dag. Er
mogen niet meer vakjes opengekrast zijn dan dat ze klachtenvrij gelopen hebben (ze
moeten namelijk met een kraskaartje waarop ze een prijs hebben, ook hun klachtenvrije
masker bij de krant inleveren).\\

 Drie dezelfde symbolen geeft een prijs van 10 euro, vier
dezelfde symbolen geeft een prijs van 25 euro, vijf dezelfde symbolen geeft een prijs van
50 euro. Bij zes dezelfde symbolen ga je in de ”bak” waaruit op het einde van de actie 12 hoofdprijzen getrokken worden: 4 van 250 euro, 4 van 500 euro en 4 van 750 euro.
Naast de zes krasvakjes is er een zevende ”troostprijsvakje”. Iemand die 6 dagen
klachtenvrij bezorgd heeft en dus 6 vakjes heeft opengekrast zonder prijs te hebben,
mag als troostprijs het zevende vakje krassen waarmee stapelpunten te verdienen zijn.
Deze stapelpunten kunnen weer gespaard worden voor de prijzen uit de spaarcatalogus.
De ”troostprijsstapelpunten”zijn onderverdeeld in 25 stapelpunten, ofwel 2,50 euro, 50
stapelpunten, ofwel 5 euro, en 75 stapelpunten, ofwel 7,50 euro.\\

Het ED gaat uit van
3000 bezorgers. Het percentage van de bezorgers dat nu klachtenvrij loopt is 75% per
dag. Men verwacht dat dit percentage gedurerende deze actie kan stijgen tot 90$\%$.
Het ED wil een totaalbedrag aan prijzen weggeven van ca. 12.300 euro. Men
realiseert zich dat je tevoren nooit precies kunt zeggen hoeveel deze actie gaat kosten en
dat e.e.a. afhankelijk is van hoe klachtenvrij gelopen wordt. Het ED wil nu weten hoe
het verder moet worden aangepakt, d.w.z., welke symbolen moeten in welke hoeveelheden
onder de hokjes worden gedrukt, opdat ongeveer het begrote totaalbedrag aan prijzen
wordt weggegeven?

\begin{comment}------------------------
\begin{itemize}
  \item Uitleg over de krasloten
  \item Uitleg over de verschillende prijzen
  \item De 14 weken en wanneer er gekrast mag worden
  \item De pechprijzen
  \item Een vermelding van de kans op een klachtvrije week
\end{itemize}
\end{comment}---------------------------

\section{Doelstelling}
De opdrachtgever, het ED, wil dat wij toezien op het te spenderen budget. Het is onze taak om ervoor te zorgen dat er ca. \euro12.300 uitgegeven wordt. Hier moeten wij voor zorgen door te bepalen welke symbolen in welke hoeveelheden onder de hokjes gedrukt moeten worden.\\

Elk kraslot heeft 6 reguliere symbolen onder de kraslaag en daarnaast een zevende vakje dat een aantal stapelpunten geeft. Deze 7 vakjes kunnen in verschrikkelijk veel configuraties worden verspreid. Denk bijvoorbeeld aan de configuratie bestaande uit 5 maantjes, dan 1 sterretje en 50 stapelpunten of de configuratie bestaande uit 2 vuurwerkpijlen, gevolgd door 2 sterretjes, 1 maantje en weer 1 vuurwerkpijl met 75 stapelpunten in het zevende vakje. Deze verschillende configuraties van krasloten hebben verschillende kansen op het winnen van een bepaald bedrag. Dit geeft ons de volgende doelstelling:\\

\begin{adjustwidth}{1.5cm}{1.5cm}
We moeten op zoek gaan naar een verdeling van de verscheidene configuraties van verspreide loten, waarbij de verwachtingswaarde van de totale uitgaven gelijk is aan \euro12.300.\\
\end{adjustwidth}

Echter ontbreekt er iets heel belangrijks aan deze doelstelling. Namelijk het principe van zekerheid. Wij willen ervoor zorgen dat de opdrachtgever gerust kan zijn dat er geen sprake is van extreme spreiding. Daarvoor komt het begrip variantie om de hoek kijken. Wij willen de opdrachtgever met een zeker percentage kunnen garanderen dat de totale uitgaven binnen een bepaalde marge rond \euro12.300 zullen uitvallen. Wij verfijnen nu de doelstelling:\\

\begin{adjustwidth}{1.5cm}{1.5cm}
We moeten op zoek gaan naar een verdeling van de verscheidene configuraties van verspreide loten, waarbij de verwachtingswaarde van de totale uitgaven gelijk is aan \euro12.300 en waarbij de variantie van de totale uitgaven zo laag mogelijk is.
\end{adjustwidth}

 

