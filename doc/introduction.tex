\chapter{Introductie}\label{cha:introduction}
In dit verslag behandelden wij een kans model voor de actie ``Knetterende krasactie'' van het Eindhovens Dagblad. \\

Met deze actie kunnen krantbezorgers prijzen winnen, door hokjes open te krassen voor iedere klachteloze dag dat zijn bezorgen. De actie duurt 14 weken, waardoor iedere bezorger 14 keer kans maakt op een prijs.\\

Het Eindhovens Dagblad (kortweg ED) heeft voor de krasactie een budget van \euro12.300,- voor de prijzen. Met ons model is het mogelijk om te berekenen hoeveel er van ieder type kraslot moet worden verdeeld om zo dicht mogelijk bij dit budget te blijven.

Het verslag is als volgt opgebouwd:
\begin{itemize}
  \item In hoofdstuk \ref{cha:scratch-off-lottery} geven we een uitgebreide uitleg van de krasactie en geven we een doelstelling
  \item In hoofdstuk \ref{cha:model} presenteren wij ons kans model
  \item In hoofdstuk \ref{cha:analysis} analyseren ons kans model om tot een goede verdeling te komen en bieden we tevens een zekerheidsanalyse
  \item In hoofdstuk \ref{cha:discussion} geven wij onze conclusie, samen met een discussie over het model en de analyse
\end{itemize}
